%!TEX root=seke.tex
% mainfile: seke.tex

\documentclass[times,10pt,twocolumn]{article}

\usepackage[table]{xcolor}% http://ctan.org/pkg/xcolor
\usepackage{latex8}
\usepackage{times}
\usepackage{pifont}

\usepackage[numbers]{natbib}
\usepackage{graphicx}
\usepackage{algorithm}
\usepackage{algorithmic}

\usepackage{tikz}
\usetikzlibrary{shapes,arrows,shadows}
\usepackage{amsmath,bm,times}
\usepackage{verbatim}



\newcommand{\goallegheny}{$^{\mbox{\footnotesize \ding{72}}}$}
\newcommand{\gosheffield}{$^{\mbox{\footnotesize \ding{73}}}$}
\newcommand{\gospace}{$\;$}

\begin{document}

\title{Empirically Evaluating the Efficiency of Search-based \\ Test Data
Generation for Relational Database Schemas\vspace*{-.1in}}

\author{Cody Kinneer \goallegheny                \and
        Gregory M.\ Kapfhammer \goallegheny      \and
        Chris Wright \gosheffield                \and
        Phil McMinn \gosheffield \vspace*{-.1in}
      }

\affiliation{
      \goallegheny \gospace Allegheny College       \and
      \gosheffield \gospace University of Sheffield
}

\maketitle

\begin{abstract}

% When evaluating an algorithm, it is often useful to speak of it's efficiency in terms of it's worst-case complexity.
% This is the case for search-based test data generation tools.
% on the search-based data generation tool \textit{SchemaAnalyst}.

% This paper introduces a framework for conducting automated empirical studies of algorithms by doubling
% the size of the input and observing the change in runtime.

% After describing a way to systematically doubling the size of structured data, we report on a study demonstrating the
% presented method's effectiveness.

The characterization of an algorithm's worst-case time complexity is useful because it succinctly captures how algorithm
runtime will grow as the input size becomes arbitrarily large.  However, for certain algorithms---such as those
performing search-based test data generation---a theoretical analysis to determine worst-case complexity is cumbersome
and thus not reported in the literature.  This paper introduces a framework that empirically determines an algorithm's
worst-case time complexity by doubling the size of the input and observing the change in runtime.  Since the relational
database is a centerpiece of modern software and the database's schema is frequently untested, we apply the doubling
technique to the domain of data generation for relational database schemas, a field where worst-case time complexities
are unknown.  In addition to demonstrating the feasibility of accurately pinpointing the worst-case runtimes of the
chosen algorithms, the results of our study reveal performance trade-offs in schema testing strategies.

\end{abstract}

\section{Introduction}

Search-based algorithms allow the application of guidance to problems.
The algorithm attempts to improve a potential solution until the
solution is acceptable. Without the use of a search-based strategy, a
problem might be approached with a random sampling or greedy
technique. In the
domain of data generation for software testing, this means that rather
than randomly selecting inputs from a program's input space, the data
generator can actively seek out qualities of an input that best fulfill
the test's goals \cite{McMinn2004a}. While this
technique has been applied to various problems, including test suite
prioritization \cite{Walcott:tsp} and testing
relational database schemas \cite{Kapfhammer2013}, 
as far as we know, no research has been done on evaluating the efficiency
of search-based test data generation. 

Because these search-based systems are complex, they are difficult to
analyze theoretically. To overcome this challenge, we attack
search-based test data generation with an empirical approach. This paper presents an empirical study of the search-based data
generation tool, called \textit{SchemaAnalyst}, which generates test suites for
relational database schemas.  To evaluate \textit{SchemaAnalyst}, we systematically double the size of the
program's input schema and record the change in its execution time.
Using this process, we explore possible configurations of
\textit{SchemaAnalyst}, revealing trade-offs in the performance of
search-based test data generation with respect to the test's goals, the
structure of the input schema, and the data generation strategy.

\begin{enumerate}
  \item A framework for automated doubling experiments
    (Section~\ref{subsec:doubling}).
  \item An empirical study evaluating the efficiency of a search-based
    data generation tool (Section~\ref{subsec:experiment}).
  \item A discussion of the trade-offs between the parameters of
    search-based test data generation and performance.  
    (Section~\ref{sec:results})

  \end{enumerate}

%!TEX root=seke.tex
% mainfile: ../seke.tex

\vspace*{-.05in}
\section{Background and Related Work}
\vspace*{-.05in}

% Do we need to talk about databases/schemas/constraints?
% if so it should go here, and we're going to have major space issues

%TODO GMK: discuss database related content
%TODO GMK: Discuss coverage criterion
%TODO GMK: Explain the subsumption hierarchy
%TODO GMK: Discuss data generators

% \subsection{Search-Based Test Data Generation}

{\bf Testing Database Schemas.} The relational database, a cornerstone of modern software, is protected by a schema that
defines integrity constraints ensuring the coherence of data. These constraints defend the schema from manipulations
that could violate requirements such as ``user names must be unique'' or ``the host name cannot be missing or unknown''.
Prior work in this area proposed coverage criteria, derived from logic coverage criteria, that establish different
levels of testing for the formulation of integrity constraints in a database schema~\cite{mcminn2015}. These range from
simple criteria that mandate the testing of successful and unsuccessful {\tt INSERT} statements into tables to more
advanced criteria that test the formulation of complex integrity constraints such as multi-column {\tt PRIMARY~KEY}s and
arbitrary {\tt CHECK} constraints. This family of criteria has been organized into a subsumption hierarchy, with
criteria such as {\em Clause-Based Active Integrity Constraint Coverage} (ClauseAICC) emerging as a stringent testing
strategy. Since space constraints limit further commentary on testing methods for database schemas, the reader is
referred to~\cite{mcminn2015} for more details.

% Since the quality of the test depends upon the characteristics of the data itself, test data generators systematically
% produce test inputs according to a criterion.

{\bf Search-Based Test Data Generation}. When testing a schema's integrity constraints for correctness, it is often
necessary to provide input to the database and observe and evaluate its execution~\cite{kapfhammer2013}.  Since the
database's behavior is dependant on the input from {\tt INSERT}s, the input space must be sufficiently explored to ensure
thorough testing.  Due to the fact that it is challenging to manually create input that supports high-quality testing,
test data generation is used to automatically produce it according to a criterion, like ClauseAICC. A search-based test
data generator is one that explores that input space using, among other components, a fitness function that rates the
data's quality, thus allowing it to improve by repeatedly searching for better inputs~\cite{mcminn2004a}.

{\bf Worst-Case Time Complexity}. A useful understanding of an algorithm's efficiency, the worst-case time complexity
gives an upper bound on how an increase in the size of the input, denoted $n$, increases the execution time of the
algorithm, $f(n)$.  This relationship is often expressed in the ``big-Oh'' notation, where $f(n)$ is $O(g(n))$ means
that the time increases by no more than on order of $g(n)$. Since the worst-case complexity of an algorithm is evident
when $n$ is large~\cite{goodrich2014}, one approach for determining the big-Oh complexity of an algorithm is to conduct
a doubling experiment with increasingly bigger input sizes. By measuring the time needed to run the algorithm on an
input of size $n$ and the time needed to run with input of size $2n$, the algorithm's order of growth can be
empirically determined~\cite{mcgeoch2012,sedgewick1998}.

The goal of a doubling experiment is to draw a conclusion regarding the efficiency of the algorithm from the ratio
$f(2n)/f(n)$ that represents the factor of change in runtime from input sizes $n$ to $2n$. For instance, a ratio of $2$
would indicate that doubling the input size resulted in the runtime's doubling, thus leading to the conclusion that the
algorithm under study is $O(n)$ or $O(n\log n)$.  Table~\ref{table:ratios} shows some common time complexities and their
corresponding ratios.

\begin{table}[t]

  \begin{center}

    \begin{tabular}{c|l}
      Ratio $f(2n)/f(n)$ & Worst-Case Conclusion              \\ \hline
      1                  & constant or logarithmic \\
      2                  & linear or linearithmic  \\
      4                  & quadratic               \\
      8                  & cubic                   \\
      % x                & $O(n^{\log x})$
    \end{tabular}

  \end{center}
  \vspace*{-.25in}

  % \caption{Worst-case time complexity conclusions that can be drawn from the doubling ratio $f(2n)/f(n)$.}\label{table:ratios}
  \caption{Conclusions for worst-case time complexity.} 
  \vspace*{-.25in}

\end{table}

{\bf Related Work}. Goldsmith et al.~\cite{Goldsmith2007} developed a tool, called \textit{Trend-Prof}, that empirically
evaluates the computational complexity of a program by using code instrumentation to count the number of times each
block of code is executed and then grouping these blocks by their behavior.  \textit{Trend-Prof} takes in a collection
of workloads, user-specified features of the workloads, and the program to be studied. While this technique results in a
more detailed analysis that the one presented in this paper, Goldsmith et al.\ did not address the issue of generating
the workloads necessary to achieve a meaningful result, which this paper's technique can handle automatically.  Our
paper is also contrasted with this prior work because it describes experiments in a domain, search-based test data
generation, where the method's worst-case time complexity is not always known.

Zhao et al.\ presented an empirical study of the performance of search-based test data generation for extended finite
state machine (EFSM) models~\cite{zhao2010}. Although this paper focused on efficiency and made preliminary observations
about the relationship between performance and the characteristics of an EFSM model, it did not, like our paper, use
doubling experiments to suggest worst-case time complexities.  Lakhotia et al.\ also reported on an experimental
analysis of the efficiency and effectiveness of search-based test data generation for C programs~\cite{lakhotia2013}.
While our paper looks at generator performance in a holistic manner, this prior work considered the number of fitness
evaluations during data generation. Similar to our use of doublers that systematically increase the size of a relational
schema, Mehrmand and Feldt empirically studied search-based data generation as the size of the program
increases~\cite{mehrmand2010}. Yet, their focus is on the generator's success in covering source code branches instead
of the efficiency of the data generation process.

% GMK NOTE: Adding in the references to the two papers suggested by PM

The empirical work presented in this paper is complemented by theoretical runtime analyses in prior research.  For
instance, Arcuri presented the first runtime analysis of a search-based test data generator called the alternating
variable method (AVM) \cite{arcuri2009}, which is also studied in this paper. Arcuri proves the worst-case time
complexity of AVM when it generates data for a simple program called ``triangle classification''. More recently, Kempka
et al.\ extended the work of Arcuri with a theoretical and empirical runtime analysis revealing that the use of certain
local search techniques with AVM yields better performance than AVM alone \cite{kempka2015}.  While our paper's
automated framework can easily be applied to new schemas---and even other types of search-based test data
generators---the results in these two aforementioned papers are more difficult to generalize.

% GMK NOTE: I think that I will not cite the Yoo et al. paper right now, due to lack of space.

% Finally, in an acknowledgement of the poor
% performance of many search-based methods, Yoo et al. explain

%!TEX root=seke.tex
% mainfile: ../seke.tex

\vspace*{-.075in}
\section{Automated Doubling Experiments}\label{sec:technique}
\vspace*{-.075in}

  \begin{figure*}
    \centering
    \newcommand{\mx}[1]{\mathbf{\bm{#1}}} % Matrix command
\newcommand{\vc}[1]{\mathbf{\bm{#1}}} % Vector command

% Define the layers to draw the diagram
\pgfdeclarelayer{background}
\pgfdeclarelayer{foreground}
\pgfsetlayers{background,main,foreground}

% Define block styles used later

\tikzstyle{sensor}=[draw, fill=blue!20, text width=5em, 
    text centered, minimum height=2.5em,drop shadow]
\tikzstyle{ann} = [above, text width=5em, text centered]
\tikzstyle{wa} = [sensor, text width=10em, fill=red!20, 
    minimum height=6em, rounded corners, drop shadow]
\tikzstyle{sc} = [sensor, text width=13em, fill=red!20, 
    minimum height=10em, rounded corners, drop shadow]

% Define distances for bordering
\def\blockdist{1.5}
\def\edgedist{2.5}

\begin{tikzpicture}
    \node (wa) [wa]  {\textit{SchemaAnalyst}};
    \path (wa.west)+(-\blockdist,-1.0) node (asr1) [sensor] {Database Schema};
    \path (asr1.west)+(-\blockdist,0.0) node (doubler) [sensor] {Schema Doubler};
    
    \path (wa.west)+(-\blockdist,1.0) node (asr2)[sensor] {Coverage Criterion};
    \path (wa.west)+(-\blockdist,0.0) node (dots)[sensor] {Data Generator}; 
    
    \path (doubler.west)+(-\blockdist,-0.0) node (doublera) [sensor] {Schema Doubler};
    \path (dots.west)+(-2.65*\blockdist,-0.0) node (dataa) [sensor] {Data Generator};
    \path (asr2.west)+(-2.65*\blockdist,-0.0) node (criteriona) [sensor] {Coverage Criterion};
    \path (doubler.west)+(-\blockdist,-1.0) node (schemaa) [sensor] {Database Schema};

 
   
    \path (wa.east)+(\blockdist,0) node (vote) [sensor] {Test Suite};
    
    \path [draw, ->] (doubler.east) -- node [above] {}
    	(asr1.180);

    \path [draw, ->] (asr1.east) -- node [above] {} 
        (wa.200);
    \path [draw, ->] (asr2.east) -- node [above] {} 
        (wa.160);
    \path [draw, ->] (dots.east) -- node [above] {} 
        (wa.180);
    \path [draw, ->] (wa.east) -- node [above] {} 
        (vote.west);
        
    \path [draw, ->] (doublera.east) -- node [above] {} 
        (doubler.west);
    \path [draw, ->] (dataa.east) -- node [above] {} 
        (dots.west);
    \path [draw, ->] (criteriona.east) -- node [above] {} 
        (asr2.west);
    \path [draw, ->] (schemaa.east) -| node [above] {} 
        (doubler.230);
        
    \path (vote.east)+(\blockdist,0) node (runtime) [sensor] {Runtime Records};
    \path [draw,->] (vote.east)+(0.3,0) -- node [above]{}
    	(runtime.west);
    \path (runtime.east)+(\blockdist,0) node (runtimeo) [sensor] {Runtime Records};

        
     \path (vote.east)+(\blockdist,-3.165) node (convalg) [sensor] {Convergence Algorithm};
     \path [draw,->] (runtime.south) -- node [above]{}
    	(convalg.north);
     \path [draw,->] (convalg.west) -| node [above,pos=.25]{Continue Experiment}
    	(doubler.290);

	\path [draw,->] (runtime.east) -- node [above]{}
    	(runtimeo);

               
    \path (wa.south) +(0,-1) node (asrs) {\textit{SchemaAnalyst} Execution};
    
    \path (asrs.south) +(0,-1.6) node (singleexp) {Experiment Manager};
  	
    \begin{pgfonlayer}{background}
        \path (doubler.west |- asr2.north)+(-0.3,0.6) node (a) {};
        \path (wa.south -| runtime.east)+(+0.3,-0.6) node (b) {};
        \path (runtime.east |- singleexp.east)+(+0.3,-0.3) node (c) {};
          
        \path[fill=yellow!20,rounded corners, draw=black!50, dashed]
            (a) rectangle (c);           
        \path (asr1.north west)+(-0.2,0.2) node (a) {};
            
    \end{pgfonlayer}
  
    \begin{pgfonlayer}{background}
        \path (asr2.west |- asr2.north)+(-0.3,0.3) node (a) {};
        \path (wa.south -| wa.east)+(+0.3,-0.3) node (b) {};
        \path (vote.east |- asrs.east)+(+0.3,-0.3) node (c) {};
          
        \path[fill=yellow!20,rounded corners, draw=black!50, dashed]
            (a) rectangle (c);           
        \path (asr1.north west)+(-0.2,0.2) node (a) {};
            
    \end{pgfonlayer}
    
    

\end{tikzpicture}

    % \vspace*{-.1in}
    \caption{Technique for conducting automatic doubling experiments.}~\label{fig:doublingexp}
    \vspace*{-.3in}
  \end{figure*}

  % GMK NOTE: We need to make sure that at some point we also talk about the automated analysis of the results.

  {\bf Overview}. The presented technique for performing automatic doubling experiments consists of two key components.
  The first is a method for systematically doubling the initially input relational schema, and the second is a rule for
  determining when a valid conclusion can be drawn from the experiment, thus allowing the doubling process to stop.

  \textbf{Doubling Schemas}. Determining worst-case complexity by a doubling experiment requires that the size of the
  input be doubled. A relational database schema is a complex artifact with many features and interrelationships.  This
  makes doubling rule implementation a non-trivial task.

  % GMK NOTE: This paragraph was too long for the reader to easily understand. Breaking it up!

  A relational database schema contains tables and columns, and constraints that restrict the values allowed into these
  entities. Since the runtime of a schema testing technique may be affected by the number of any of these, it is
  desirable to have a strategy for doubling each one. Doubling the number of tables or columns in a schema is relatively
  easy.  It is possible to double the number of tables in a schema by following this rule: for every table present in
  the schema, create a new empty table. It is important that the new tables be empty to avoid changing more than one
  doubling parameter at once---if the new tables contained columns, for instance, then the number of tables and columns
  in the schema both would be increased, thus interfering with assessing table doubling's impact on
  performance.  Additionally, doubling the number of columns can be accomplished by, for every table in the schema, and
  for every column, adding a new column to that table.

  % GMK NOTE: I think that this point could lead to confusion. Since it is not essential, I am removing it for now.

  % An alternative plan could be starting with a very large number of tables with no primary keys that could accommodate
  % doubling their number for many trials, but this introduces unnecessary performance concerns.

  Doubling integrity constraints is more challenging.  The {\tt FOREIGN KEY} constraint, for instance, denotes a relationship
  between two tables, thus making it difficult to double without introducing extraneous database entities or cyclic
  dependencies.  Since a {\tt CHECK} constraint can express arbitrary conditions, it is also challenging to double if the
  meaning of each constraint must be considered to ensure satisfiability.  Since a table can only contain one {\tt
  PRIMARY KEY}, if a schema contains five tables, then at most it can have five {\tt PRIMARY KEY} constraints, as adding more
  keys would require creating more tables, which should be avoided.

  % GMK QUESTION: So, **what are** the constraints that can be doubled in this fashion? This must be clarified.
  %CBK ANSWER: NOT NULL, UNIQUE, and CHECK

  % GMK NOTE: I recognize that we always used this phrase. But, does it really add that much to the paper?
  % CBK NOTE: No I suppose not

  % We refer to this doubling strategy as, semantic doubling.

  Because of these issues, and others like them, we focus our attention on constraints that can be doubled as follows:
  for every table and for every constraint, duplicate that constraint and re-add it to the table.  Constraints such as
  {\tt NOT NULL}, {\tt UNIQUE}, and {\tt CHECK} are amenable to doubling in this fashion.  It is worth noting that
  constraints doubled this way would not have an impact on what data the schema would allow or disallow into a database,
  since they amount to a restatement of existing constraints.  However, since the goal is to evaluate performance, the
  results should not be affected as long as the test data generation technique must still process and consider these
  additional constraints.

  \textbf{Automatic Experimentation}. To determine worst-case complexity, an input $n$ is doubled until the ratio $f(2n)
  / f(n)$ converges to a stable value.  To account for random error, every time $n$ is doubled, $f(n)$ is computed ten
  times and the median time is used for calculating the ratios; we chose the median to minimize the effect of outliers.
  If the mean is used instead, then a single abnormally long run could have an outsized impact on the result.
  Figure~\ref{fig:doublingexp} shows the overall structure of the experimentation framework.

  Convergence checking is necessary because of the fact that worst-case time is only evident for large values of $n$.
  If too few doubles are tried, then the experiment may terminate before $n$ reaches a value where the true worst-case
  time complexity is apparent. At the same time, for inefficient  algorithms, each additional doubling run incurs a
  substantial time overhead. For the sake of efficiency, the experiment should terminate as quickly as possible.

  % CBK NOTE: Introducing the new equation for ratio with respect to time is helpful for explaining the convergence
  % algorithm, but seems like a restatement of the ratio equation.  I considered only using this new equation when
  % discussing the ratio for consistency, but it is unnecessarily complicated for earlier content

  To test for convergence, for every time $t$, where $t$ denotes the number of times the input has been doubled, we
  record the each doubling ratio $r_t = \frac{f(2^t n)}{f(2^{t-1}n)}$. The current ratio $r_c$ is compared to a previous
  ratio $r_p$ where $p$ is determined by a $\mathit{lookback}$ value, such that $p=c-\mathit{lookback}$.  The result of
  the comparison is a $\mathit{difference}$ value, given by $\mathit{difference} = |r_c - r_p|$.  This is then compared
  to a $\mathit{tolerance}$ value, and the experiment is judged to have converged when $\mathit{difference}<\mathit{tolerance}$.
  The $\mathit{lookback}$ and $\mathit{tolerance}$ values are both configured before the experiment is run.

  Another consequence of worst-case time only being apparent for large $n$, is that a very small initial $n$ may appear
  to converge to one, which indicates constant time complexity. To prevent the experiment from incorrectly terminating
  given a small starting $n$, our method requires that a program under study display a ratio of one for a $\mathit{minimum}$
  number of times before judging that the ratio does in fact converge to one.  That is, if $r_c = 1$, $t >
  \mathit{minimum}$ must be true in addition to the tolerance test before the experiment is declared convergent.  The
  $\mathit{minimum}$ parameter is also configured before an experiment.  Because a doubling ratio of one signifies
  constant or logarithmic time complexity, requiring these doubles does not significantly increase the time needed to
  run the experiment, while providing assurance that a small ratio is not due to an insufficiently small $n$.

%!TEX root=../seke.tex
% mainfile: ../seke.tex

\section{Empirical Analysis}

\textbf{Experimental Design}. To gain a full picture of the performance trade-offs, we conducted an experiment for every configuration
of the parameter space i.e. (schema, coverage criterion, and data generator, and doubling technique). 

In our experimental study, we set $\mathit{tolerance}$ to $0.40$ and $\mathit{lookback}$ to $4$. This value was chosen
by performing doubling experiments on various algorithms with known worst case time complexities, and observing that the
ratio converged to the correct value with this configuration.  We also conducted preliminary experiments in an attempt
to select good parameter values. We set $\mathit{minimum}$ to $20$ after observing that \textit{SchemaAnalyst} stopped
displaying constant behavior after around 5 doubles.  Preliminary studies showed that while experiments for fast
configurations could be completed in less than an hour, slower configurations required days.  Since there are over four
thousand possible configurations, the study requires a substantial amount of computational resources.  As a solution, we
conducted the experiments on a high-performance computing (HPC) cluster containing 195 worker nodes of various hardware
configurations, ranging from 12 to 16 CPU cores and 24 to 256 GB of memory, and using the 64-bit Redhat operating
system.

% GMK NOTE: It would be better to call this a tree model instead of a regression tree -- the word regression is also
% used in the software testing literature to have a different meanining.

% Regression tree
% Belongs with results_trees, but must be here for page placement within the paper


\section{Results}
  \label{sec:results}

Our experiments reveal that the number of tables in the schema has the
greatest impact on the runtime of \textit{SchemaAnalyst}. The BigOh
complexity\dots 
%TODO need to analyze BigOh

To gain a more nuanced understanding of the results, we construct a
regression tree predicting runtime by the predictors \texttt{Tables,
Columns, Uniques, NotNulls, Checks, Criterion, and DataGenerator}. The
package \textit{ctree} for the R language was used to produce the tree,
shown as Figure~\ref{fig:atree}. The regression tree confirms that the
number of tables has the largest impact on runtime, as can be seen by
the fact that the node 1 splits on the number of tables, and the
significant difference between nodes 6 and 7, which are also
distinguished by the number of tables according to node 5. The tree also
reveals that when the number of tables in the schema is small, the
choice of coverage criterion is the most important predictor for
runtime.  This is shown by node 2, however, the nodes resulting from this
prediction, nodes 3 and 4, do not seem very distinct.  

To gain more insight into the behavior of \textit{SchemaAnalyst} when
the number of tables is small, a new tree was constructed with the same
parameters, with the exception that \texttt{Tables} was removed from
the list of predictors. The resulting regression tree is shown as
Figure~\ref{fig:ttree}.  Node 1 in the new tree also indicates that
the choice of criterion has the largest impact when the number of tables
is not considered.  Node 2 shows that the next most significant
predictor is the data generator, and node 3 shows the next most
significant factor is the number of columns in the schema.  The
differences between the leaves of the tree however, are still not
readily apparent. 

According to the decisions produced by the regression tree, the choice
of coverage criterion and data generator have an impact on the runtime
of \textit{SchemaAnalyst}. To show the effect of these choices, we
present Figure~\ref{fig:crites}, which shows the effect of coverage
criterion on runtime, and Figure~\ref{fig:datas}, which shows the effect
of data generator on runtime.  

For coverage criteria, the most apparent pattern is that AUCC,
ClauseAICC, and CondAICC seem to cause runtime to increase by about the
same amount, with the other criterions taking roughly the same amount of
time.

For data generator, the Random and Random defDults generators took the
most amount of time by a distinctive margin, and a less pronounced
hierarchy between AVM, AVM defaults, Directed Random, and Directed
Random Defaults can be observed.  

While the box and whisker plots allow us to see how choices between
coverage criterions and data generators affects runtime, the question
remains if these differences are statistically and practically
significant. To answer this question, we present the Wilcoxon rank sum
test and the $\hat{A}_{12}$ test.  

%TODO talk about what the tests mean, how to interpret results here

We perform these tests for every pair
of coverage criterions, and every pair of data generators.
Table~\ref{tab:crites} shows the results of these tests for coverage
criteria, and Table~\ref{tab:datas} shows the results for data
generators.

\begin{figure*}
\centering
  \centering
  \includegraphics[width=.75\linewidth]{../diagrams/AllTree.pdf}
  \caption{Regression tree using all variables to predict runtime in
  minutes. \vspace{-.15in}}
  \label{fig:atree}
  \vspace{-.15in} 
\end{figure*}

\begin{figure*}
\centering
  \centering
  \includegraphics[width=.75\linewidth]{../diagrams/NoTableCtreesd.pdf}
  \caption{Regression tree predicting runtime excluding Tables.\vspace{-.15in}}
  \label{fig:ttree}
  \vspace{-.15in} 
\end{figure*}


\begin{figure}
\centering
  \centering
  \includegraphics[width=1\linewidth]{../diagrams/CriterionvsTime.pdf}
  \caption{Coverage criterion versus runtime in minutes.\vspace{-.15in}}
  \label{fig:crites}
  \vspace{-.15in} 
\end{figure}

\begin{figure}
\centering
  \centering
  \includegraphics[width=1\linewidth]{../diagrams/DataGeneratorvsTime.pdf}
  \caption{Data generator versus runtime in minutes.\vspace{-.15in}}
  \label{fig:datas}
  \vspace{-.15in} 
\end{figure}


\begin{table*}[h]
\begin{tabular}{llllllllll}
           & APC      & ANCC     & CondAICC & NCC      & AUCC     & AICC     & ClauseAICC & ICC      & UCC   \\ 
APC        & NA       & 0.425    & 0.337    & 0.484    & 0.334    & 0.413    & 0.329      & 0.481    & 0.449 \\
ANCC       & 2.20E-16 & NA       & 0.407    & 0.561    & 0.405    & 0.484    & 0.399      & 0.554    & 0.526 \\
CondAICC   & 2.20E-16 & 2.20E-16 & NA       & 0.671    & 0.503    & 0.581    & 0.492      & 0.656    & 0.634 \\
NCC        & 1.20E-02 & 2.20E-16 & 2.20E-16 & NA       & 0.335    & 0.417    & 0.322      & 0.491    & 0.461 \\
AUCC       & 2.20E-16 & 2.20E-16 & 6.92E-01 & 2.20E-16 & NA       & 0.577    & 0.490      & 0.651    & 0.628 \\
AICC       & 2.20E-16 & 1.70E-02 & 2.20E-16 & 2.20E-16 & 2.20E-16 & NA       & 0.412      & 0.571    & 0.547 \\
ClauseAICC & 2.20E-16 & 2.20E-16 & 2.72E-01 & 2.20E-16 & 1.40E-01 & 2.20E-16 & NA         & 0.662    & 0.641 \\
ICC        & 4.00E-03 & 2.20E-16 & 2.20E-16 & 1.83E-01 & 2.20E-16 & 2.20E-16 & 2.20E-16   & NA       & 0.472 \\
UCC        & 9.30E-16 & 3.83E-05 & 2.20E-16 & 7.36E-10 & 2.20E-16 & 5.73E-13 & 2.20E-16   & 9.29E-06 & NA    \\ 
\end{tabular}
\caption{For each pair of coverage criterions, lower left shows Wilcox
Rank Sum Test, upper right shows $\hat{A}_{12}$.}
\label{tab:crites}
\end{table*}


\begin{table*}[h]
\begin{tabular}{lllllll}
                         & Random   & Random Defaults & Directed Random & Directed Random Defaults & AVM      & AVM Defaults \\
Random                   & NA       & 0.538           & 0.740           & 0.789                    & 0.627    & 0.680        \\
Random Defaults          & 6.59E-13 & NA              & 0.673           & 0.701                    & 0.564    & 0.617        \\
Directed Random          & 2.20E-16 & 2.20E-16        & NA              & 0.543                    & 0.360    & 0.435        \\
Directed Random Defaults & 2.20E-16 & 2.20E-16        & 9.74E-16        & NA                       & 0.328    & 0.395        \\
AVM                      & 2.20E-16 & 2.20E-16        & 2.20E-16        & 2.20E-16                 & NA       & 0.572        \\
AVM Defaults             & 2.20E-16 & 2.20E-16        & 2.20E-16        & 2.20E-16                 & 2.20E-16 & NA          
\end{tabular}
\caption{For each pair of Data Generators, lower left shows Wilcox Rank
Sum Test, upper right shows $\hat{A}_{12}$.}
\label{tab:datas}
\end{table*}


\subsection*{Threats to Validity}

Our technique for doubling the number of check constraints on the schema
is simply to duplicate the existing check constraints. It is possible
that \textit{SchemaAnalyst} does less work processing these copied check
constraints than it would given unique check constraints. However,
doubling the check constraints in this way is an easy to implement,
semantically significant way of evaluating \textit{SchemaAnalyst}.

Additionally, since worst-case time is only apparent for large $n$, 
it is possible that the experiment terminated too quickly.  To guard 
against this problem, Algorithms~\ref{alg:convergence} and~\ref{alg:tuning}
were tested on various other algorithms with known worst-case complexities, and 
found to be reliable.

%\section{Related Work}

  %TODO GMK: We need more of this
  % Even if you know of some papers, you can send some to me and I can read them

  Goldsmith et al. \cite{Goldsmith:2007:MEC:1287624.1287681} 
  developed a system to empirically evaluate computational
  complexly.  Their system, \textit{Trend-Prof}, uses code
  instrumentation to count the number of times each block 
  of code is executed, and then
  groups these blocks by their behavior.  \textit{Trend-Prof} takes in a
  collection of workloads, user specified features of the workloads, and
  the program to be studied. This technique results in a more powerful
  analysis. However, the authors do not address the
  issue of generating the workloads necessary to achieve a meaningful
  result, and we attempt to do this automatically.  Additionally, our
  approach is novel because we apply it to a domain where the
  theoretical scalability is not yet known.

%%!TEX root=../seke.tex
% mainfile: ../seke.tex

\section{Conclusions and Future Work}

The automated doubling experiment was able to determine the worst case time complexity of \textit{SchemaAnalyst} with
respect to the number of check constraints in the input schema, for the \textsc{constraintCACCoverage} criterion and the
\textsc{directedRandom} data generator.  Additional experiments will be conducted on other criteria and data generators.
Additionally, other factors that may influence the runtime of schema analysis, such as the number of primary keys,
foreign keys, tables, columns, etc will be investigated.


\bibliographystyle{IEEEtran}
\bibliography{seke.bib}

\end{document}
