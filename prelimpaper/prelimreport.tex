% This is "sig-alternate.tex" V2.0 May 2012
% This file should be compiled with V2.5 of "sig-alternate.cls" May 2012
%
% This example file demonstrates the use of the 'sig-alternate.cls'
% V2.5 LaTeX2e document class file. It is for those submitting
% articles to ACM Conference Proceedings WHO DO NOT WISH TO
% STRICTLY ADHERE TO THE SIGS (PUBS-BOARD-ENDORSED) STYLE.
% The 'sig-alternate.cls' file will produce a similar-looking,
% albeit, 'tighter' paper resulting in, invariably, fewer pages.
%
% ----------------------------------------------------------------------------------------------------------------
% This .tex file (and associated .cls V2.5) produces:
%       1) The Permission Statement
%       2) The Conference (location) Info information
%       3) The Copyright Line with ACM data
%       4) NO page numbers
%
% as against the acm_proc_article-sp.cls file which
% DOES NOT produce 1) thru' 3) above.
%
% Using 'sig-alternate.cls' you have control, however, from within
% the source .tex file, over both the CopyrightYear
% (defaulted to 200X) and the ACM Copyright Data
% (defaulted to X-XXXXX-XX-X/XX/XX).
% e.g.
% \CopyrightYear{2007} will cause 2007 to appear in the copyright line.
% \crdata{0-12345-67-8/90/12} will cause 0-12345-67-8/90/12 to appear in the copyright line.
%
% ---------------------------------------------------------------------------------------------------------------
% This .tex source is an example which *does* use
% the .bib file (from which the .bbl file % is produced).
% REMEMBER HOWEVER: After having produced the .bbl file,
% and prior to final submission, you *NEED* to 'insert'
% your .bbl file into your source .tex file so as to provide
% ONE 'self-contained' source file.
%
% ================= IF YOU HAVE QUESTIONS =======================
% Questions regarding the SIGS styles, SIGS policies and
% procedures, Conferences etc. should be sent to
% Adrienne Griscti (griscti@acm.org)
%
% Technical questions _only_ to
% Gerald Murray (murray@hq.acm.org)
% ===============================================================
%
% For tracking purposes - this is V2.0 - May 2012

\documentclass{sig-alternate}
\usepackage{algorithm}
\usepackage{algorithmic}
\begin{document}
%
% --- Author Metadata here ---
% \conferenceinfo{QSIC}{'14 Dallas, Texas USA}
%\CopyrightYear{2007} % Allows default copyright year (20XX) to be over-ridden - IF NEED BE.
%\crdata{0-12345-67-8/90/01}  % Allows default copyright data (0-89791-88-6/97/05) to be over-ridden - IF NEED BE.
% --- End of Author Metadata ---

\title{Empirically Evaluating the Efficiency of Search-based Test Suite
Generation for Relational Database Schemas}
%
% You need the command \numberofauthors to handle the 'placement
% and alignment' of the authors beneath the title.
%
% For aesthetic reasons, we recommend 'three authors at a time'
% i.e. three 'name/affiliation blocks' be placed beneath the title.
%
% NOTE: You are NOT restricted in how many 'rows' of
% "name/affiliations" may appear. We just ask that you restrict
% the number of 'columns' to three.
%
% Because of the available 'opening page real-estate'
% we ask you to refrain from putting more than six authors
% (two rows with three columns) beneath the article title.
% More than six makes the first-page appear very cluttered indeed.
%
% Use the \alignauthor commands to handle the names
% and affiliations for an 'aesthetic maximum' of six authors.
% Add names, affiliations, addresses for
% the seventh etc. author(s) as the argument for the
% \additionalauthors command.
% These 'additional authors' will be output/set for you
% without further effort on your part as the last section in
% the body of your article BEFORE References or any Appendices.

\numberofauthors{3} %  in this sample file, there are a *total*
% of EIGHT authors. SIX appear on the 'first-page' (for formatting
% reasons) and the remaining two appear in the \additionalauthors section.
%
\author{
% You can go ahead and credit any number of authors here,
% e.g. one 'row of three' or two rows (consisting of one row of three
% and a second row of one, two or three).
%
% The command \alignauthor (no curly braces needed) should
% precede each author name, affiliation/snail-mail address and
% e-mail address. Additionally, tag each line of
% affiliation/address with \affaddr, and tag the
% e-mail address with \email.
%
% 1st. author
\alignauthor
Cody Kinneer\\
       \affaddr{Allegheny College}\\
       \email{kinneerc@allegheny.edu}
% 2nd. author
\alignauthor
Luke Smith\\
       \affaddr{Allegheny College}\\
       \email{smithl4@allegheny.edu}
% 3rd. author
\alignauthor 
Gregory Kapfhammer\\
       \affaddr{Allegheny College}\\
       \email{gkapfham@allegheny.edu}
}
% There's nothing stopping you putting the seventh, eighth, etc.
% author on the opening page (as the 'third row') but we ask,
% for aesthetic reasons that you place these 'additional authors'
% in the \additional authors block, viz.
% \additionalauthors{Additional authors: John Smith (The Th{\o}rv{\"a}ld Group,
% email: {\texttt{jsmith@affiliation.org}}) and Julius P.~Kumquat
% (The Kumquat Consortium, email: {\texttt{jpkumquat@consortium.net}}).}
% \date{30 July 1999}
% Just remember to make sure that the TOTAL number of authors
% is the number that will appear on the first page PLUS the
% number that will appear in the \additionalauthors section.

\maketitle
\begin{abstract}
When evaluating an algorithm, it is often useful to speak of it's
efficiency in terms of it's worst case complexity.  However, for certain
cases such as search-based algorithms, determining an algorithm's
efficiency by theoretical analysis is unfeasible.  This paper introduces a
framework for conducting automated empirical studies of algorithms by
doubling the size of the input and observing the change in execution
time. This method is then applied to the domain of data generation for
relational database schemas. A technique for systematically doubling the
size of schemas was implemented, and an empirical study was conducted on
the search-based data generation tool \textit{SchemaAnalyst}. For the
parameters of \textit{SchemaAnalyst} tested,  The study
concluded that \textit{SchemaAnalyst} was $O(n^2)$ with respect to the
number of check constraints in the input schema.
\end{abstract}

%A category including the fourth, optional field follows...
\category{D.2.8}{Software Engineering}{Metrics}[complexity measures, performance measures]

\section{Introduction}
Search-based algorithms allow guidance to be applied to problems that
might otherwise be approached with a random sampling technique. In the
domain of data generation for software testing, this means that rather
than randomly selecting inputs from a program's input space, the
qualities of an input that best fulfill the test's goals can be
actively sought out by the data generator \cite{McMinn:Thesis}. While this
technique has been applied to various problems, including test suite
prioritization \cite{Walcott:tsp} and testing
relational database schemas \cite{Kapfhammer2013}, 
as far as we know, no research has been done on evaluating the efficiency
of search-based test data generation. 

This paper presents an empirical study of the search-based data
generation tool \textit{SchemaAnalyst}, which generates test suites for
relational database schemas.  To evaluate \textit{SchemaAnalyst}, a tool
was implemented in Java to systematically double the size of the
program's input and record the change in it's execution time. Using this
technique, for the coverage criterion and data generator tested, 
\textit{SchemaAnalyst} was found to be $O(n^2)$ with respect
to the number of check constraints in the schema. The contributions of
this paper are therefore as follows:
\begin{enumerate}
  \item A framework for automated doubling experiments
  \item An empirical study evaluating the efficiency of a search-based
    data generation tool
  \end{enumerate}

\section{Background}

Worst case time complexity is a useful measure of an algorithms
efficiency, or how increasing the size
of the input $n$ increases the execution time of the algorithm, $f(n)$.
This relationship is often expressed in big-Oh notation, where $f(n)$
is $O(g(n))$ means that the time increases by order of $g(n)$. The worst
case complexity of an algorithm is evident when $n$ is large 
\cite{Goodrich:Data}. One approach for determining the big-Oh complexity
of an algorithm $f$ is to conduct a doubling experiment. By measuring the
time needed to run the algorithm on $n$, and the time needed to run on $2n$, the order of growth
of $f$ can be determined \cite{Mcgeoch:Algorithmics,
Sedgewick:Analysis}. 

Intuitively, the goal of a doubling experiment is to draw a conclusion
regarding the efficiency of the algorithm from the ratio
$f(2n)/f(n)$. This ratio represents the factor of change in runtime from
input $n$ to $2n$. A ratio of $2$ would indicate that doubling the
input resulted in runtime doubling. We could then conclude that the
algorithm under study is $O(n)$. 

\section{Technique}
To determine worst case complexity, an input $n$ was doubled until the 
ratio $f(2n) / f(n)$ converged to a stable value. To account for random
error, every time $n$ was doubled, $f(n)$ was recorded ten times
, and the median time was used for calculating the ratios. The overall 
structure of the experiment is shown in Algorithm~\ref{alg:main}.

This convergence checking is necessary because of the fact that worst
case time is only apparent for large values of $n$. If too few doubles
are tested, then the experiment may terminate before $n$ reaches a value
where the worst case time is apparent. At the same time, for inefficient 
algorithms, each additional double tested incurs a substantial time cost, 
so to conduct the testing efficiently the experiment should terminate as 
quickly as possible.

To test for convergence, the last four ratios were compared, and the
sum of differences between them is compared to a tolerance value. A
tolerance value of $0.40$ was chosen 
by performing doubling experiments on various algorithms with known worst case time
complexities, and observing that the ratio converged to the correct
value when $\mathit{diff} < 0.40$. The convergence algorithm is shown
as Algorithm~\ref{alg:convergence}.
  
Another consequence of worst case time only being apparent for large
$n$, is that a very small initial $n$ may appear to converge to $1$,
which indicates constant or logarithmic time. To prevent the
experiment from incorrectly terminating given a small starting $n$, we
require that an algorithm under test display a ratio of $1$ for many
runs before judging that the ratio does in fact converge to $1$. In this case,
the experiment is considered convergent if the ratio remains $1$ for 
twenty consecutive doubles.  Because $1$ signifies constant or logarithmic 
time, requiring these doubles does not significantly increase the time needed
to run the experiment, while providing assurance that a small ratio is not due to
an insufficiently small $n$. This test is shown as Algorithm~\ref{alg:tuning}.

\begin{algorithm}[t]
    \caption{Run Doubling Experiment}
    \begin{algorithmic}
      \WHILE{Diff not convergent || N not large enough}
        \FOR{$\mathit{count} < 10$}
        \STATE Run Test
        \STATE $\mathit{count}++$
        \ENDFOR
        \STATE Double Schema
        \ENDWHILE
    \end{algorithmic}
    \label{alg:main}
  \end{algorithm}

 \begin{algorithm}[t]
    \caption{Diff not Convergent}
    \begin{algorithmic}
      \STATE $\mathit{diff} = |(r_4 - r_3) + (r_3 -r_2) + (r_2 - r_1)|$
      \IF{$\mathit{diff} < 0.40$}
      \RETURN FALSE
      \ELSE
      \RETURN TRUE
      \ENDIF
    \end{algorithmic}
    \label{alg:convergence}
  \end{algorithm}

 \begin{algorithm}[t]
    \caption{N not Large Enough}
    \begin{algorithmic}
      \IF{$\mathit{ratio} \approx 1$}
      \IF{$\mathit{Doubles} < 20$}
      \STATE minRuns++
      \RETURN TRUE
      \ENDIF
      \ENDIF
      \RETURN FALSE
    \end{algorithmic}
    \label{alg:tuning}
  \end{algorithm}

\section{Experimental Design}
To analyze \textit{SchemaAnalyst}, the iTrust and NistWeather case
studies provided by \textit{SchemaAnalyst} were used as the initial 
input schemas.
Both of these schemas are taken from real world applications. The factor $n$
under study was the number of check constraints present on the schema.  A tool was
implemented to double the number of these constraints. Generating
synthetic check constraints is non-trivial because there are many
possible check constraints, and generating a constraint that is
unsatisfiable might cause the data generation tool to take a longer
amount of time than should be the case. To avoid this problem, we
instead duplicate the existing check constraints present on the schema
rather than attempt to generate new ones. This technique is easy
to implement, and ensures that the check constraints added are
semantically valid.  For every table in the input schema, the tool 
duplicated the existing check constraints and added the duplicates 
to the table.  

The test suite generation tool provided by \textit{SchemaAnalyst}
requires a coverage criterion and a data generator to be specified. A
coverage criterion is a system of rules that generate test requirements
\cite{Ammann:Testing}. The data generator is the object that generates
the test data according to the rules specified by the coverage
criterion. The criterion used in the experiment was
\textsc{constraintCACCoverage}, 
the data generator was \textsc{directedRandom}. The
testing tools were implemented in Java, and were both compiled and run using
version 1.7 of the compiler and JVM. The experiment was executed on an Ubuntu 13.10 machine with a 2.4
GHz quad core CPU running the 3.11.0-18-generic x86\_64 GNU/Linux
kernel.

\section{Results}
Our technique was successfully able to determine the worst case time
complexity of \textit{SchemaAnalyst} with regard to the number of check
constraints on the input schema, using the
\textsc{constraintCACCoverage} coverage criterion and the
\textsc{directedRandom} data generator. Under these conditions, 
\textit{SchemaAnalyst} displayed $O(n^2)$ behavior. Figures~\ref{fig:iTrust} 
and~\ref{fig:NistWeather} show the data collected during the experiment with a
quadratic fit. The x axis shows by what factor the number of initial
check constraints $i$ have been increased. For $x = 100$, the number of check
constraints on the schema is $100i$. The y axis shows the time in
nanoseconds \textit{SchemaAnalyst} needed to generate a test suite for
the input schema. Each point on the graph represents the amount of time
\textit{SchemaAnalyst} ran for a schema with $ix$ check constraints.
The equation shown is the best fit quadratic equation for that data.  A
quadratic relationship indicates that \textit{SchemaAnalyst} is
$O(n^2)$.  The value $r^2$ is a measure of the quality of fit between a model and
the data. The closer $r^2$ is to $1$, the better the quality of the
model.  An $r^2$ of $.997$ and $1$ indicates that the models are a good fit for
the data.

\begin{figure*}
\centering
  \centering
  \includegraphics[width=.5\linewidth]{iTrustChecks.pdf}
  \caption{Time vs Check Constraints on iTrust.}
  \label{fig:iTrust}
\end{figure*}
\begin{figure*}
  \centering
  \includegraphics[width=.5\linewidth]{NistWeatherChecks.pdf}
  \caption{Time vs Check Constraints on NistWeather.}
  \label{fig:NistWeather}
\end{figure*}

\subsection*{Threats to Validity}
Our technique for doubling the number of check constraints on the schema
is simply to duplicate the existing check constraints. It is possible
that \textit{SchemaAnalyst} does less work processing these copied check
constraints than it would given unique check constraints. However,
doubling the check constraints in this way is an easy to implement,
semantically significant way of evaluating \textit{SchemaAnalyst}.

Additionally, since worst case time is only apparent for large $n$, 
it is possible that the experiment terminated too quickly.  To guard 
against this problem, Algorithms ~\ref{alg:convergence} and ~\ref{alg:tuning}
were tested on various other algorithms with known worst case complexities, and 
found to be reliable.

\section{Future Work}
The automated doubling experiment was able to determine the worst case
time complexity of \textit{SchemaAnalyst} with respect to the number of
check constraints in the input schema, for the
\textsc{constraintCACCoverage} criterion and the
\textsc{directedRandom} data generator.  Additional experiments will be
conducted on other criterions and data factories. Additionally, other factors 
that may influence the runtime of schema analysis,
such as the number of primary keys, foreign keys, tables, columns, etc
will be investigated.

%\end{document}  % This is where a 'short' article might terminate

%
% The following two commands are all you need in the
% initial runs of your .tex file to
% produce the bibliography for the citations in your paper.
\bibliographystyle{abbrv}
\bibliography{sigproc}  % sigproc.bib is the name of the Bibliography in this case
% You must have a proper ".bib" file
%  and remember to run:
% latex bibtex latex latex
% to resolve all references
%
% ACM needs 'a single self-contained file'!
%
%APPENDICES are optional
%\balancecolumns
\end{document}
