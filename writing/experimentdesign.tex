%!TEX root=../seke.tex
% mainfile: ../seke.tex

\begin{table}[t]
  \centering

  {\footnotesize
  \begin{tabular}{r | c c c}
                           Schema & Tables & Columns & Constraints \\ \hline
    BioSQL                        & 28     & 129     & 186 \\
    Cloc                          & 2      & 10      & 0 \\
    iTrust                        & 42     & 309     & 134 \\
    JWhoisServer                  & 6      & 49      & 50 \\
    NistWeather                   & 2      & 9       & 13 \\
    NistXTS748                    & 1      & 3       & 3 \\
    NistXTS749                    & 1      & 3       & 3 \\
    RiskIt                        & 13     & 57      & 36 \\
    UnixUsage                     & 8      & 32      & 24
\end{tabular}}

  \vspace*{-.05in}
  \caption{Database schemas used in the experiments.}~\label{tab:schemas}
  \vspace*{-.25in}

\end{table}

\vspace{-.05in}
\section{Empirical Analysis}
\vspace{-.05in}

\textbf{Experimental Design}. To gain a full picture of the performance trade-offs, we conducted an experiment for every
configuration of the parameter space (i.e., schema, coverage criterion, data generator, and doubling technique).
Table~\ref{tab:schemas} shows that the experiments focused on nine schemas that contain between 1 and 42 distinct
tables, 3 to 309 columns, and up to 186 constraints. Including all of the test adequacy criteria proposed by McMinn et
al.~\cite{mcminn2015}, the experiments study ``weak'' criteria (i.e., APC, NCC, ICC, and UCC), ``moderately strong''
ones (i.e., ANCC, AICC, and AUCC), and ``strong'' criteria (i.e., CondAICC and ClauseAICC). More details about each
criterion, including its formal definition and relationship to the other criteria, are available in~\cite{mcminn2015}. We
used all six test data generators provided by the {\em SchemaAnalyst} tool for automated test data
generation~\cite{kapfhammer2013}, with four techniques employing a variant of random search and two based on Korel's
alternating variable method. After a restart of the search for suitable test data, all of these data generators could
start with either default or random values.

% GMK NOTE: Can we safely cut this sentence? I don't think that it is absolutely needed (I have already revised it)

% In an effort to ensure good picks for parameters, we also conducted preliminary experiments.

In our study, we set $\mathit{tolerance}$ to $0.40$ and $\mathit{lookback}$ to $4$. These values were chosen by
performing doubling experiments on various algorithms, with known worst-case time complexities, and observing that the
ratio converged to the correct value with this configuration.  After observing that \textit{SchemaAnalyst} stopped
displaying constant behavior after around 5 doubles, we set
$\mathit{minimum}$ to be four times this number.
Preliminary studies showed that, while experiments for ``fast'' configurations could be completed in less than an hour,
``slower'' configurations required days.  Since there are over two thousand possible configurations, the study needed
a substantial amount of computational resources.  As a solution, we ran the experiments on a high-performance computing
(HPC) cluster containing 195 worker nodes of various hardware configurations, ranging from 12 to 16 CPU cores and 24 to
256 GB of memory, and using a 64-bit GNU/Linux operating system.

% GMK NOTE: It would be better to call this a tree model instead of a regression tree -- the word regression is also
% used in the software testing literature to have a different meanining.

% Regression tree
% Belongs with results_trees, but must be here for page placement within the paper
