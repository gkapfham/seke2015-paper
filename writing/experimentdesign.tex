%!TEX root=../seke.tex
% mainfile: ../seke.tex

\section{Experimental Design}

To gain a full picture of the performance trade-offs, we conducted an experiment for every configuration of the
parameter space i.e. (schema, coverage criterion, and data generator, and doubling technique). A coverage criterion is a
system of rules that generate test requirements~\cite{Ammann2008}. The data generator is the object that generates the
test data according to the rules produced by the coverage criterion~\cite{Ammann2008}.

%TODO explain generator and coverage criterion in background instead

In our experimental study, we set $\mathit{differanceTolerance}$ to
$0.40$. This value was chosen by performing doubling
experiments on various algorithms with known worst case time
complexities, and observing that the ratio converged to the correct
value when $\mathit{diff} < 0.40$.
We set $\mathit{doublesTolerance}$ to $20$ after observing that
\textit{SchemaAnalyst} stopped displaying constant behavior after around
5 doubles.

Preliminary studies showed that while experiments for
fast configurations could be completed in less than an
hour, slower configurations required days. Since there
are over four thousand possible configurations, the study
requires a substantial amount of computational resources.
As a solution, we conducted the experiments on the Iceberg
HPC Cluster at the University of Sheffield, containing
195 worker nodes of various hardware configurations,
ranging from 12 to 16 CPU cores and 24 to 256 GB of
memory, and use a 64-bit Redhat based operating system.
