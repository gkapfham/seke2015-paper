\section{Experimental Design}

In our experimental study, we set $\mathit{differanceTolerance}$ to
$0.40$. This value was chosen by performing doubling
experiments on various algorithms with known worst case time
complexities, and observing that the ratio converged to the correct
value when $\mathit{diff} < 0.40$.
We set $\mathit{doublesTolerance}$ to $20$ after observing that
\textit{SchemaAnalyst} stopped displaying constant behavior after around
5 doubles.

To analyze \textit{SchemaAnalyst}, the iTrust, NistWeather, and BioSQL case
studies provided by \textit{SchemaAnalyst} were used as the initial 
input schemas.
The factor $n$
under study was the number of tables, columns, and constraints on the schema.  
Generating
synthetic constraints is non-trivial because there are many
possible constraints, and generating a constraint that is
unsatisfiable might cause the data generation tool to take a longer
amount of time than should be the case. To avoid this problem, we
instead duplicate the existing constraints present on the schema
rather than attempt to generate new ones. This technique is easy
to implement and ensures that the constraints added are
semantically valid.  For every table in the input schema, the tool 
duplicated the existing constraints and added the duplicates 
to the table.  

The test suite generation tool provided by \textit{SchemaAnalyst}
requires a coverage criterion and a data generator to be specified. A
coverage criterion is a system of rules that generate test requirements
\cite{Ammann:Testing}. The data generator is the object that generates
the test data according to the rules specified by the coverage
criterion.
The data generator used in this experiment was \textsc{avs}. We ran the
experiment for every combination of schema, coverage criterion, and doubling
technique. The
doubling experimentation software was implemented in Java, and were both compiled and run using
version 1.7 of the compiler and Java Virtual Machine. The experiment was
executed on the Iceberg HPC Cluster at the University of Sheffield.
