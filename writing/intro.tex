\section{Introduction}

Search-based algorithms allow the application of guidance to problems.
The algorithm attempts to improve a potential solution until the
solution is acceptable. Without the use of a search-based strategy, a
problem might be approached with a random sampling or greedy
technique. In the
domain of data generation for software testing, this means that rather
than randomly selecting inputs from a program's input space, the data
generator can actively seek out qualities of an input that best fulfill
the test's goals \cite{McMinn2004a}. While this
technique has been applied to various problems, including test suite
prioritization \cite{Walcott:tsp} and testing
relational database schemas \cite{Kapfhammer2013}, 
as far as we know, no research has been done on evaluating the efficiency
of search-based test data generation. 

This paper presents an empirical study of the search-based data
generation tool, called \textit{SchemaAnalyst}, which generates test suites for
relational database schemas.  To evaluate \textit{SchemaAnalyst}, a tool
was implemented in Java to systematically double the size of the
program's input and record the change in it's execution time. Using this
technique, for the coverage criterion and data generator tested, 
\textit{SchemaAnalyst} was found to be $O(n^2)$ with respect
to the number of check constraints in the schema. The contributions of
this paper are therefore as follows:

\begin{enumerate}
  \item Techniques for systematically doubling relational database schemas
    (Section~\ref{subsec:experiment})
  \item A framework for automated doubling experiments
    (Subsection~\ref{subsec:doubling})
  \item An empirical study evaluating the efficiency of a search-based
    data generation tool (Section~\ref{sec:results})
  \end{enumerate}
