%!TEX root=../seke.tex
% mainfile: ../seke.tex

\vspace*{-.25in}
\section{Introduction}
\vspace*{-.05in}

% GMK NOTE: Removing the reference to the Walcott paper, not absolutely critical.

% Search-based algorithms allow the application of guidance to problems.
% The algorithm attempts to improve a potential
% solution until it is acceptable according to a fitness function.

Many disciplines, such as science, finance, and medicine, rely on relational databases to maintain large amounts of
critical information~\cite{kapfhammer2013}. The relational database schema defines the structure of a database and
protects the integrity of the data. This makes testing the database schema necessary to avoid the corruption of data.
Search-based algorithms, that use a fitness function to offer guidance to the correct solution, have been applied to
this challenging problem~\cite{kapfhammer2013}. Although data generation for relational schemas may also be solved,
albeit less effectively, with random generation techniques~\cite{mcminn2015}, the use of search-based approaches ensures
that data creation methods can actively seek out test inputs that best fulfill testing goals~\cite{mcminn2004a}.

% GMK NOTE: Revised this content in the paragraph above and below

% Without the use of a search-based strategy, a problem might be approached with a random sampling or greedy technique.
% In the domain of data generation for software testing, this means that rather than randomly selecting inputs from a
% program’s input space, the data generator can actively seek out qualities of an input that best fulfills the test’s
% goals~\cite{McMinn2004a}.

Despite the effectiveness of search-based data generation methods, there is, to the best of our knowledge, little prior
research that fully studies their efficiency and characterizes their worst-case time complexity. In part, we attribute
this dearth of past work to the fact that these systems are complex, thus making a generalizable theoretical analysis
hard.

In response to the lack of insight into the performance of search-based methods, this paper presents a fully automated
performance evaluation framework that employs doubling experiments to suggest worst-case time complexities and
classification and regression trees to identify efficiency trends. Applying this framework to the automated performance
evaluation of search-based test data generation for database schemas, the results reveal trade-offs in efficiency with
respect to the chosen testing goals, the structure of the relational schema, and the data generation strategy.

Since the presented approach is fully automated, it enabled a comprehensive study suggesting the worst-case time
complexity of all the relevant data generator configurations. Although this paper focuses on automatically evaluating
efficiency of search-based test data generation for database schemas, the presented technique is general and thus
applicable to a wide range of methods using heuristic search. In summary, this paper's important \mbox{contributions
include}:

% GMK NOTE: I captured the essence of this content and then made room for another key point

% attack the performance evaluation of search-based test data generation with an empirical
% approach. We present an empirical performance study of the search-based test data generation tool, called
% \textit{SchemaAnalyst}, which generates test suites for relational database schemas. To evaluate worst-case performance,
% we developed a technique for automatically conducting doubling experiments. Using this process, we explored
% \textsc{xmany} configurations of \textit{SchemaAnalyst}, revealing
% Our automatic technique enabled a comprehensive empirical study, that would otherwise have been infeasible, and the
% results have important practical significance for the selection of parameters in search-based test data generation
% tools.

\begin{enumerate}
  \itemsep0in

  \item A performance evaluation framework that automatically conducts and analyzes the results from doubling
    experiments with search-based methods.

  \item With a systematic focus on a wide variety of configurations, an empirical study revealing trade-offs
    in search-based test data generation for relational schemas.

  \item Empirically derived suggestions for the worst-case time complexity of search-based test data generators.

  \end{enumerate}
  \vspace*{-.15in}
