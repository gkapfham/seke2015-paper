%!TEX root=../seke.tex
% mainfile: ../seke.tex

\vspace*{-.25in}
\section{Introduction}

% GMK NOTE: Removing the reference to the Walcott paper, not absolutely critical.

% Search-based algorithms allow the application of guidance to problems. 
% The algorithm attempts to improve a potential
% solution until it is acceptable according to a fitness function. 

Many disciplines, such as science, finance, and medicine, rely on relational databases to maintain large amounts of
critical information~\cite{kapfhammer2007}. The relational database schema defines the structure of a database and
protects the integrity of the data. This makes testing the database schema necessary to avoid the corruption of data.
Search-based algorithms, that use a fitness function to offer guidance to the correct solution, have been applied to
this challenging problem~\cite{Kapfhammer2013}. Although data generation for relational schemas may also be solved,
albeit less effectively, with random generation schemes~\cite{McMinn2015}, the use of search-based methods ensures that
data creation methods can actively seek out test inputs that best fulfill testing goals~\cite{McMinn2004a}.

% Without the use of a search-based strategy, a problem might be approached with a random sampling or greedy technique.
% In the domain of data generation for software testing, this means that rather than randomly selecting inputs from a
% program’s input space, the data generator can actively seek out qualities of an input that best fulfills the test’s
% goals~\cite{McMinn2004a}.

Despite the effectiveness of search-based techniques, there is no research regarding their efficiency reported in the
literature. Because these search-based systems are complex, they are often difficult to analyze theoretically. To
overcome this challenge, we attack the performance evaluation of search-based test data generation with an empirical
approach. We present an empirical performance study of the search-based test data generation tool, called
\textit{SchemaAnalyst}, which generates test suites for relational database schemas. To evaluate worst-case performance,
we developed a technique for automatically conducting doubling experiments. Using this process, we explored
\textsc{xmany} configurations of \textit{SchemaAnalyst}, revealing trade-offs in the performance of search-based test
data generation with respect to the test’s goals, the structure of the input schema, and the data generation strategy.
Our automatic technique enabled a comprehensive empirical study, that would otherwise have been infeasible, and the
results have important practical significance for the selection of parameters in search-based test data generation
tools.

\begin{enumerate}
  \item A framework for automated doubling experiments
    (Section~\ref{subsec:doubling}).
  \item An empirical study evaluating the efficiency of a search-based
    data generation tool (Section~\ref{subsec:experiment}).
  \item A discussion of the trade-offs between the parameters of
    search-based test data generation and performance.  
    (Section~\ref{sec:results})
  \end{enumerate}
