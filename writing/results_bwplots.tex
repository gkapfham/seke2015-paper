% vim: ft=tex
%!TEX root=seke.tex
% mainfile: ../seke.tex

While the trees provide insight into the relative impact of each predictor, the box and whisker plots shown in the
leaves of the trees do not furnish a detailed view of the choices within each predictor.  To gain a finer-grained
understanding, we created box and whisker plots of our own:  Figure~\ref{fig:crites} shows the influence of coverage
criterion on runtime, while Figure~\ref{fig:datas} shows the effect of data generator on runtime.
Figure~\ref{fig:crites} shows that the strongest coverage criteria in the subsumption hierarchy (i.e., AUCC, ClauseAICC,
and CondAICC) cause runtime to increase the most, followed by ANCC and AICC, and then the remaining criteria (i.e., APC
through UCC).  We anticipate that the stronger criteria always lead to higher time overheads because they force {\em
SchemaAnalyst} to generate more tests.  Also, criteria at the same level in the hierarchy engender similar runtimes.

Figure~\ref{fig:datas} reveals that, by a substantial margin, the Random and Random Defaults generators took the most
time to generate data. This counterintuitive result suggests that less effective data generators actually take longer to
create data than those that are known to be more effective~\cite{kapfhammer2013}.  A less pronounced difference between
the remaining generators can be observed, with the use of default values consistently being faster than the use of
random values at restart.

\begin{table*}[th]
\vspace*{-.2in}
  \centering
  \small
\begin{tabular}{llllllllll}
           & APC      & ANCC     & CondAICC & NCC      & AUCC     & AICC     & ClauseAICC & ICC      & UCC   \\
APC        & NA       & \cellcolor{gray!25}0.425    &
\cellcolor{gray!45}0.337    & 0.484    &\cellcolor{gray!45} 0.334    &
\cellcolor{gray!25}0.413    &\cellcolor{gray!45} 0.329      & 0.481    & 0.449 \\
ANCC       & 2.20E-16 & NA       &\cellcolor{gray!25} 0.407
&\cellcolor{gray!25} 0.561    &\cellcolor{gray!25} 0.405    & 0.484
&\cellcolor{gray!25} 0.399      & 0.554    & 0.526 \\
CondAICC   & 2.20E-16 & 2.20E-16 & NA       &\cellcolor{gray!45} 0.671
& 0.503    &\cellcolor{gray!25} 0.581    & 0.492
&\cellcolor{gray!45} 0.656    &\cellcolor{gray!25} 0.634 \\
NCC        & 1.20E-02 & 2.20E-16 & 2.20E-16 & NA
&\cellcolor{gray!45} 0.335    &\cellcolor{gray!25} 0.417
&\cellcolor{gray!45} 0.322      & 0.491    & 0.461 \\
AUCC       & 2.20E-16 & 2.20E-16 & \fbox{6.92E-01} & 2.20E-16 & NA       &
\cellcolor{gray!25}0.577    & 0.490      &\cellcolor{gray!45} 0.651
&\cellcolor{gray!25} 0.628 \\
AICC       & 2.20E-16 & 1.70E-02 & 2.20E-16 & 2.20E-16 & 2.20E-16 & NA
&\cellcolor{gray!25} 0.412      &\cellcolor{gray!25} 0.571    & 0.547 \\
ClauseAICC & 2.20E-16 & 2.20E-16 & \fbox{2.72E-01} & 2.20E-16 & 1.40E-01 &
2.20E-16 & NA         & \cellcolor{gray!45}0.662    &\cellcolor{gray!45} 0.641 \\
ICC        & 4.00E-03 & 2.20E-16 & 2.20E-16 &\fbox{1.83E-01} & 2.20E-16 & 2.20E-16 & 2.20E-16   & NA       & 0.472 \\
UCC        & 9.30E-16 & 3.83E-05 & 2.20E-16 & 7.36E-10 & 2.20E-16 &
5.73E-13 & 2.20E-16   & 9.29E-06 & NA    \\ \hline
& Rank-Sum: & significant & \fbox{insignificant} & &
$\hat{A}_{12}$: & none & \cellcolor{gray!25} small &
\cellcolor{gray!45} medium & \cellcolor{gray!65} large
\end{tabular}

\caption{For each pair of coverage criteria, lower left shows Wilcoxon Rank-Sum Test, upper right shows $\hat{A}_{12}$.}
\label{tab:crites}
\vspace*{-.2in}
\end{table*}
