%!TEX root=seke.tex
% mainfile: ../seke.tex

While the box and whisker plots allow us to see how choices between coverage criteria and data generators affects
runtime, the question remains if these differences are statistically and practically significant. To answer this
question, we calculate the Wilcoxon rank-sum test and the $\hat{A}_{12}$ test.

We perform these tests for every pair of coverage criteria, and every pair of data generators.  Table~\ref{tab:crites}
shows the results of these tests for coverage criteria, and Table~\ref{tab:datas} shows the results for data generators.

The Wilcoxon rank-sum test is a non-parametric test for hypothesis testing.  If the result of the test is greater than
the significance level ($0.05$ is frequently used), then the collections are indistinguishable.  If however, the result
is less, then the collections are differentiable.  The $\hat{A}_{12}$ test is similar, but for drawing conclusions about
the practical difference between two collections of data.  A result $a=0.5$0 means that any difference is not practically
significant, while $a>0.56$ or $a<0.44$ signifies a small difference, $a>0.64$ or $a<0.36$ denotes a medium difference,
and $a>0.71$ or $a<0.29$ indicates a large difference.

Table~\ref{tab:crites} reveals that changing the criterion results in statistically significant differences in runtimes,
with the exception of changing between four constraints at the top of the subsumption hierarchy, and two constraints at
the bottom.  The $\hat{A}_{12}$ results generally show a small to medium practical effect of switching between criteria
at the high or low end of the hierarchy, and small or no effect of switching between constraints at the same level of the
hierarchy.

% This table might be on the chopping block

Table~\ref{tab:datas} shows a statistically significant difference between all pairs of data generators. The
$\hat{A}_{12}$ results all choices in data generator have at least a small practical impact, with the exception of
choosing between random and random defaults, and directed random and directed random defaults.  Change between these
sets of data generators results in a large to medium effect, and comparing either AVM data generator to another
primarily resulted in a small difference.

\begin{table*}[h]
  \centering
  \small
\begin{tabular}{lllllll}
                         & Random   & Random Defaults & Directed Random & Directed Random Defaults & AVM      & AVM Defaults \\
Random                   & NA       & 0.538
&\cellcolor{gray!65} 0.740           &\cellcolor{gray!65} 0.789
&\cellcolor{gray!25} 0.627    &\cellcolor{gray!45} 0.680        \\
Random Defaults          & 6.59E-13 & NA              &
\cellcolor{gray!45}0.673           &\cellcolor{gray!45} 0.701
&\cellcolor{gray!25} 0.564    & \cellcolor{gray!25}0.617        \\
Directed Random          & 2.20E-16 & 2.20E-16        & NA
& 0.543                    &\cellcolor{gray!25} 0.360    &\cellcolor{gray!25} 0.435        \\
Directed Random Defaults & 2.20E-16 & 2.20E-16        & 9.74E-16
& NA                       &\cellcolor{gray!45} 0.328    &\cellcolor{gray!25} 0.395        \\
AVM                      & 2.20E-16 & 2.20E-16        & 2.20E-16
& 2.20E-16                 & NA       &\cellcolor{gray!25} 0.572        \\
AVM Defaults             & 2.20E-16 & 2.20E-16        & 2.20E-16        & 2.20E-16                 & 2.20E-16 & NA
\end{tabular}
\caption{For each pair of Data Generators, lower left shows Wilcoxon Rank-Sum Test, upper right shows $\hat{A}_{12}$.}
\label{tab:datas}
\end{table*}
