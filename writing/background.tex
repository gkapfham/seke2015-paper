%!TEX root=seke.tex
% mainfile: ../seke.tex

\vspace*{-.05in}
\section{Background and Related Work}
\vspace*{-.05in}

% Do we need to talk about databases/schemas/constraints?
% if so it should go here, and we're going to have major space issues

%TODO GMK: discuss database related content
%TODO GMK: Discuss coverage criterion
%TODO GMK: Explain the subsumption hierarchy
%TODO GMK: Discuss data generators

% \subsection{Search-Based Test Data Generation}

{\bf Testing Database Schemas.} The relational database, a cornerstone of modern software, is protected by a schema that
defines integrity constraints ensuring the coherence of data. These constraints defend the schema from manipulations
that could violate requirements such as ``user names must be unique'' or ``the host name cannot be missing or unknown''.
Prior work in this area proposed coverage criteria, derived from logic coverage criteria, that establish different
levels of testing for the formulation of integrity constraints in a database schema~\cite{mcminn2015}. These range from
simple criteria that mandate the testing of successful and unsuccessful INSERT statements into tables to more advanced
criteria that test the formulation of complex integrity constraints such as multi-column PRIMARY KEYs and arbitrary
CHECK constraints. This family of criteria has been organized into a subsumption hierarchy, with criteria such as {\em
Clause-Based Active Integrity Constraint Coverage} (ClauseAICC) emerging as a stringent testing strategy. Since space
constraints limit further commentary on testing methods for database schemas, the reader is referred
to~\cite{mcminn2015} for more details.

% Since the quality of the test depends upon the characteristics of the data itself, test data generators systematically
% produce test inputs according to a criterion.

{\bf Search-Based Test Data Generation}. When testing a schema's integrity constraints for correctness, it is often
necessary to provide input to the database and observe and evaluate its execution~\cite{kapfhammer2013}.  Since the
database's behavior is dependant on the input from INSERTs, the input space must be sufficiently explored to ensure
thorough testing.  Due to the fact that it is challenging to manually create input that supports high-quality testing,
test data generation is used to automatically produce it according to a criterion, like ClauseAICC. A search-based test
data generator is one that explores that input space using, among other components, a fitness function that rates the
data's quality, thus allowing it to improve by repeatedly searching for better inputs~\cite{mcminn2004a}.

{\bf Worst-Case Time Complexity}. A useful understanding of an algorithm's efficiency, the worst-case time complexity
gives an upper bound on how an increase in the size of the input, denoted $n$, increases the execution time of the
algorithm, $f(n)$.  This relationship is often expressed in the ``big-Oh'' notation, where $f(n)$ is $O(g(n))$ means
that the time increases by no more than on order of $g(n)$. Since the worst-case complexity of an algorithm is evident
when $n$ is large~\cite{goodrich2014}, one approach for determining the big-Oh complexity of an algorithm is to conduct
a doubling experiment with increasingly bigger input sizes. By measuring the time needed to run the algorithm on an
input of size $n$ and the time needed to run with input of size $2n$, the algorithm's order of growth can be
empirically determined~\cite{mcgeoch2012,sedgewick1998}.

The goal of a doubling experiment is to draw a conclusion regarding the efficiency of the algorithm from the ratio
$f(2n)/f(n)$ that represents the factor of change in runtime from input sizes $n$ to $2n$. For instance, a ratio of $2$
would indicate that doubling the input size resulted in the runtime's doubling, thus leading to the conclusion that the
algorithm under study is $O(n)$ or $O(n\log n)$.  Table~\ref{table:ratios} shows some common time complexities and their
corresponding ratios.

{\bf Related Work}. Goldsmith et al.~\cite{Goldsmith2007} developed a tool, called \textit{Trend-Prof}, that empirically
evaluates the computational complexity of a program by using code instrumentation to count the number of times each
block of code is executed and then grouping these blocks by their behavior.  \textit{Trend-Prof} takes in a collection
of workloads, user-specified features of the workloads, and the program to be studied. While this technique results in a
more detailed analysis that the one presented in this paper, Goldsmith et al.\ do not address the issue of generating
the workloads necessary to achieve a meaningful result, which this paper's technique can handle automatically.  Our
paper is also contrasted with this prior work because it describes experiments in a domain, search-based test data
generation, where the method's worst-case time complexity is not known.

Zhao et al.\ present an empirical study of the performance of search-based test data generation for extended finite
state machine (EFSM) models~\cite{zhao2010}. Although this paper does focus on efficiency and make preliminary
observations about the relationship between performance and the characteristics of an EFSM model, it does not, like our
paper, use doubling experiments to suggest worst-case time complexities. Lakhotia et al. also present an experimental
analysis of the efficiency and effectiveness of search-based test data generation for C programs~\cite{lakhotia2013}.
While our paper looks at generator performance in a holistic manner, this prior work only considers the number of
fitness evaluations during data generation. Similar to our use of doublers that systematically increase the size of a
relational schema, Mehrmand and Feldt study search-based test data generation as the size of the
program under test increases~\cite{mehrmand2010}. Yet, their focus is on the generator's success in covering source code
branches instead of the efficiency of the data generation process.

% GMK NOTE: I think that I will not cite the Yoo et al. paper right now, due to lack of space.

% Finally, in an acknowledgement of the poor
% performance of many search-based methods, Yoo et al. explain

\begin{table}[t]

  \begin{center}

    \begin{tabular}{c|l}
      Ratio $f(2n)/f(n)$ & Worst-Case Conclusion              \\ \hline
      1                  & constant or logarithmic \\
      2                  & linear or linearithmic  \\
      4                  & quadratic               \\
      8                  & cubic                   \\
      % x                & $O(n^{\log x})$
    \end{tabular}

  \end{center}
  \vspace*{-.15in}

  \caption{Time complexity conclusions that can be drawn from the doubling ratio.}\label{table:ratios}
  \vspace*{-.30in}

\end{table}
