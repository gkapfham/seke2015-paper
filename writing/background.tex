\section{Background}

% Do we need to talk about databases/schemas/constraints?
% if so it should go here, and we're going to have major space issues
% TODO GMK: discuss database related content

\subsection{Search-Based Test Data Generation}
When testing a system for correctness, it is often necessary to
provide input to the system in order to observe and evaluate its
execution. Additionally, when the system's behavior is dependant on the
input, the input space must be sufficiently explored to ensure thorough
testing.  Obtaining enough input data to support high quality testing can
be challenging, so test data generation is used to automatically
produce this data. Since the quality of the test depends upon the
quality of the data, test data generation tools systematically produce data
according to some requirement or criterion. A search-based test data 
generation tool is one that explores that data space using a fitness
function.  The fitness function rates the quality of the data, and allows
the generator to try and improve by searching for higher quality test data.

%TODO GMK: Discuss coverage criterions
%TODO GMK: Explain the subsumpion hierarchy

%TODO GMK: Discuss data generators


\subsection{Worst Case Time Complexity}

Worst-case time complexity is a useful measure of an algorithms
efficiency, or how increasing the size
of the input $n$ increases the execution time of the algorithm, $f(n)$.
Worst-case refers to the efficiency in the worst possible scenario.
This relationship is often expressed in big-Oh notation, where $f(n)$
is $O(g(n))$ means that the time increases by no more than on order of $g(n)$. The
worst-case complexity of an algorithm is evident when $n$ is large 
\cite{Goodrich:Data}. One approach for determining the big-Oh complexity
of an algorithm is to conduct a doubling experiment. By measuring the
time needed to run the algorithm on an input $n$, and the time needed to
run on $2n$, the order of growth of the algorithm can be determined \cite{Mcgeoch:Algorithmics,Sedgewick:Analysis}. 

Intuitively, the goal of a doubling experiment is to draw a conclusion
regarding the efficiency of the algorithm from the ratio
$f(2n)/f(n)$. This ratio represents the factor of change in runtime from
input $n$ to $2n$. A ratio of $2$ would indicate that doubling the
input resulted in runtime doubling. We could then conclude that the
algorithm under study is $O(n)$ or $O(n\log n)$.
Table~\ref{table:ratios} shows some common time complexities and the
corresponding ratios.

\begin{table}[h]
\begin{tabular}{l|l}
Ratio & Conclusion              \\ \hline
1     & Constant or Logarithmic \\
2     & Linear or Linearithmic  \\
4     & Quadratic               \\
8     & Cubic                   \\
x     & $O(n^{\log x})$          
\end{tabular}
\label{table:ratios}
\caption{Conclusions regarding efficiency that can be drawn from the
doubling ratio.}
\end{table}
