%!TEX root=seke.tex
% mainfile: ../seke.tex

To gain a more nuanced understanding of the results, we constructed a regression tree using the \textit{ctree} package
for the R language. A regression tree attempts to use the values of predictor variables to predict the value of a
response variable. A regression tree accomplishes this by repeatedly splitting the data according to what predictor
variable has the most influence on the response variable. Each node in the tree represents a choice of predictor
variable, and the level of the node indicates its importance to the prediction---higher being more important.

\textit{ctree} produced the tree shown as Figure~\ref{fig:atree} to predict the runtime of \textit{SchemaAnalyst} with
the predictor variables: tables, columns, {\tt UNIQUE}s, {\tt NOT NULL}s, {\tt CHECK}s, chosen criterion, and the data
generator.  The regression tree confirms that the number of tables has the greatest impact on runtime, and also reveals
that when the number of tables in the schema is small, the choice of coverage criterion is most significant.

While tables had a large impact when the number of tables was over 197 thousand, in practice schemas are unlikely to be
this large. Another invocation of \textit{ctree}, excluding tables from the list of predictors, provided insight into
the behavior of \textit{SchemaAnalyst} for more practical table sizes. Coverage criterion emerged as the most important
predictor for runtime, followed by the choice of data generator, and then the number of columns in the schema.
