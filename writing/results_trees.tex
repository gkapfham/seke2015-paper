% vim: ft=tex
%!TEX root=seke.tex
% mainfile: ../seke.tex

To gain a more nuanced understanding of the results, our tool constructed a conditional inference tree model using the
\textit{ctree} package in the R language. These trees use the values of predictor variables (e.g., the adequacy
criterion) to model the value of a response variable (e.g., {\em SchemaAnalyst}'s runtime);~\textit{ctree} accomplishes this
by repeatedly splitting the data according to what predictor variable has the most influence on the response variable.
Each tree node represents a choice of predictor variable, and the level of the node indicates its importance to
the prediction, with higher nodes being more important to predicting generation time.

% \textit{ctree} produced the tree shown as Figure~\ref{fig:atree} to predict the runtime of \textit{SchemaAnalyst} with
% the predictor variables: tables, columns, {\tt UNIQUE}s, {\tt NOT NULL}s, {\tt CHECK}s, chosen criterion, and the data
% generator.  The regression tree confirms that the number of tables has the greatest impact on runtime, and also reveals
% that when the number of tables in the schema is small, the choice of coverage criterion is most significant.

% \textit{ctree} produced the tree shown as Figure~\ref{fig:atree} to predict the runtime of \textit{SchemaAnalyst} with

Using predictor variables for the number of tables, columns, {\tt UNIQUE}s, {\tt NOT NULL}s, {\tt CHECK}s, the chosen
criterion, and the data generator, \textit{ctree} produced the tree model in Figure~\ref{fig:atree}.  In addition to
confirming that the number of tables has the greatest impact on runtime, the tree also reveals that, when the number of
tables in the schema is small, the choice of coverage criterion is most significant.  While tables had a large impact
when the number of tables was over $197,000$, in practice schemas are unlikely to be this large. Another invocation of
\textit{ctree}, excluding tables from the list of predictors, provided insight into the behavior of
\textit{SchemaAnalyst} for more practical table sizes. In this tree, not shown due to space constraints, the coverage
criterion emerged as the most important predictor for runtime, followed by the choice of data generator, and then the
number of columns.
