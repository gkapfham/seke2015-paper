%!TEX root=seke.tex
% mainfile: ../seke.tex

% Box and whisker plots
% Belongs to results_bwplots, but must be here for positioning reasons

\begin{figure*}
\centering
\begin{subfigure}{0.5\textwidth}
  \centering
  \includegraphics[width=1\linewidth]{diagrams/CriterionvsTime.pdf}
  \caption{Coverage criterion versus runtime in minutes.}
  \label{fig:crites}
\end{subfigure}%
\begin{subfigure}{0.5\textwidth}
  \centering
  \includegraphics[width=1\linewidth]{diagrams/DataGeneratorvsTime.pdf}
  \caption{Data generator versus runtime in minutes.}
  \label{fig:datas}
\end{subfigure}
\label{fig:bwplots}
\caption{Box and whisker plots for criterion and data
  generator.}
\end{figure*}

To gain a more nuanced understanding of the results, we construct a
regression tree using the \textit{ctree} package for the R langauge. A
regression tree attempts to use the values of predictor variables to
predict the value of a responce variable. A regression tree accopmlishes
this by repeadidily splitting the data according to what predictor varaible has the
most influcence on the response varaible. Each node in the tree represents a choice
of predictor varaible, and the level of the node indicates its
importance to the prediction---higher being more important.

\textit{ctree} produced the tree shown as Figure~\ref{fig:atree} to
predict the runtime of \textit{SchemaAnalyst} with the predictor varaibles:
tables, columns, UNIQUEs, NOT NULLs, CHECKs, Criterion, and
DataGenerator. The regression tree confirms that the number of
tables has the largest impact on runtime, and also reveals that when the number of tables in the schema is small, 
the choice of coverage criterion is the most important predictor for runtime.

Another invocation of \textit{ctree}, excluding tables from the list of
predictors, confirmed that when the number of tables in the schema is small, the choice of coverage criterion is the
most important predictor for runtime, followed by the choice of data
generator, and then the number of columns in the schema.
