%!TEX root=../seke.tex
% mainfile: ../seke.tex

\section{Conclusions and Future Work}

% The automated doubling experiment was able to determine the worst case time complexity of \textit{SchemaAnalyst} with
% respect to the number of check constraints in the input schema, for the \textsc{constraintCACCoverage} criterion and the
% \textsc{directedRandom} data generator.  Additional experiments will be conducted on other criteria and data generators.
% Additionally, other factors that may influence the runtime of schema analysis, such as the number of primary keys,
% foreign keys, tables, columns, etc will be investigated.

This paper presented an automated method for empirically suggesting the worst-case time complexity of search-based test
data generation methods. Focusing on the domain of relational database schemas, our approach repeatedly doubles the size
of the input schema and observes the commensurate change in runtime. Although some results are inconclusive, we find
that, in many cases, data generation is linear or linearithmic and, in others, it is cubic, quadratic, or worse.  Our
automated method also revealed that, for all of the test adequacy criteria in the subsumption hierarchy presented by
McMinn et al.~\cite{mcminn2015}, stronger criteria always necessitate more time for test data generation.

Since this paper's technique did not consider the doubling of constraints like {\tt FOREIGN KEY}s, future work will
focus on creating doublers for these unstudied constraints. Additionally, the current doubling mechanism avoids
introducing semantically invalid constraints by restating existing constraints; in future work we plan to implement and
evaluate more realistic ways to double relational schemas. Because certain experiments timed out before converging, we
also want to re-run these configurations with longer time limits and more memory. Finally, we will investigate how
automated parameter tuning can support choosing the convergence condition without manual study before running
experiments in a new computational environment. Ultimately, the combination of the presented framework with the
completed future work will yield an effective way to empirically understand the worst-case case time complexity
of search-based test data generation techniques.
